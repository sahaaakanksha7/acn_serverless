\documentclass[letterpaper,twocolumn,10pt]{article}
\usepackage{epsfig,xspace,url}
\usepackage{authblk}


%\usepackage[breaklinks]{hyperref}

\title{EMARS: Efficient Management and Allocation of Resources in Serverless\\}
\author{Aakanksha Saha, Aisha Syed, Jacobus Van der Merwe, Sonika Jindal}
\date{}
\affil{University of Utah}

\begin{document}

\maketitle
\title{\textbf{Abstract:}} The latest serverless
solutions are really server-hidden and built to host functions 
and hide that how the functions runs on servers or how scaling is 
done. Unlike SaaS or PaaS that are always running, but scale 
on-demand, serverless workloads run on-demand, and consequently, 
scale on-demand. There are mutual economic pressure as Cloud 
providers need to cost-compete by using datacenters resources more 
efficiently while Cloud customers seek to reduce cost by minimizing resource wastage. Both can be satisfied by better matching of application needs to allocated services. 
In this paper, we explore and compare various serverless platforms in terms of latencies incurred by them because of resource allocation. We analyse various serverless platforms and propose benchmark for performance measurements. 

\section{Introduction}
\label{sec:first}
Serverless cloud computing is a paradigm in which the cloud providers dynamically provisions machine resources for the user applications to run. And in turn, the user pays only for the compute resources utilized by their application. The applications are broken into modular functions which are stateless, event-driven and short lived.

The serverless architecture comes in useful for cases like bursty workloads where demand is not known upfront. Such scenarios require quick expansion and shrinking of resources. The need can be achieved if the functions can be spawned quickly in containers. To further reduce the invocation latencies, there is a need to efficiently manage the containers such that the startup time ($\approx$ 100ms) of containers can be avoided.

The following are the key challenges in a serverless platform concerning resource management:
\begin{enumerate}
\item {\it Cost:} Minimizing resource usage by serverless function, both when it is executing and when idle. 

\item {\it Cold start:} A key differentiator of serverless is the ability to scale to zero, or not
charging customers for idle time. Scaling to zero, however, leads to the problem
of cold starts, and paying the penalty of getting serverless code ready to run.
Techniques to minimize the cold start problem while still scaling to zero are
critical.

\item {\it Resource limits:} Resource limits are needed to ensure that the platform can handle
load spikes, and manage attacks. Enforceable resource limits on a serverless
function include memory, execution time, bandwidth, and CPU usage. In additional,
there are aggregate resource

\item {\it Scaling:} The platform must ensure the scalability and elasticity of users’ functions. This includes pro-actively provisioning resources in response to load, and
in anticipation of future load.
\end{enumerate}

Proactive allocation of resources remains a challenging problem. One of the solutions could be to train the system for predicting the requirements based upon the data gathered. The useful information can be in the form of type of the requests received for certain period, type of resources needed to handle the request etc. In a nutshell, resource management can be done based upon the container type and application type by appropriately allocating the system resources.

In this paper, we make the following contributions:
\begin{enumerate}
\item Analysed various serverless platforms for latency in function invocations.
\item Build a benchmark for evaluating the performance.
\item Propose EMARS for efficient resource allocation. 
\end{enumerate}

\section{Implementation}
\label{sec:third}
\begin{enumerate}
\item Current handling of containers in OpenLambda
\item Proposed container allocation in OpenLambda
\item Flow diagram with EMARS approach.
\end{enumerate}

\section{Evaluation}
\label{sec:fourth}
\begin{enumerate}
\item Test against the benchmarking suite
\item Compare performance with current platforms.
\end{enumerate}


\section{Future work}
\label{sec:fifth}
\begin{enumerate}
\item Apply machine learning to predict the application behaviour collected from logs and manage resources appropriately.
\item Strong isolation of containers running on the shared platform to enhance security.
\end{enumerate}

\section{Conclusion}
\label{sec:sixth}
In this work we present an analysis of resource allocation done inside various serverless platforms. We also present the effect of resource allocation techniques used in some open source serverless platforms. Further we propose a design based upon OpenLambda for resource allocation as per the learnings from variety of loads.

\section{Related work}
\label{sec:seventh}
There have been work done in analysing the latencies due to container allocation \cite{mcgrath.wp}. To enable the serverless technology, there are still some gaps to be filled in terms of monitoring and analysis which is done by various tools. There are tools like \cite{iopipe} to monitor the function performance specific to AWS Lambda. OpenLambda \cite{openlambda.wp} compares the response times of lambda functions vs elastic BS virtual machines.

To our knowledge, most of the platforms perform resource scaling using eager pooling and resource shrinking like in OpenWhisk\cite{whisk} and Fission.io\cite{fission}. An initial pool of containers is allocated and based upon the demand the pool is expanded or shrunk.

%Some benchmarking has also been done for performance of serverless framework\cite{faasperf.wp}.

{
  %\footnotesize 
  \small 
  \bibliographystyle{acm}
  \bibliography{biblio}
}
\end{document}



